% Generated by Sphinx.
\def\sphinxdocclass{report}
\documentclass[a4paper,12pt,oneside]{sphinxmanual}
\usepackage[utf8]{inputenc}
\DeclareUnicodeCharacter{00A0}{\nobreakspace}
\usepackage{cmap}
\usepackage[T1]{fontenc}
\usepackage[francais]{babel}
\usepackage{times}
\usepackage[Sonny]{fncychap}
\usepackage{longtable}
\usepackage{sphinx}
\usepackage{multirow}


\title{Développement Web: Conception et développement d'une plateforme de téléversement d'images en version mobile}
\date{27 mars 2015}
\release{Collège du Sud}
\author{Daniel Filipe Nunes Silva}
\newcommand{\sphinxlogo}{}
\renewcommand{\releasename}{Travail de maturié}
\makeindex

\makeatletter
\def\PYG@reset{\let\PYG@it=\relax \let\PYG@bf=\relax%
    \let\PYG@ul=\relax \let\PYG@tc=\relax%
    \let\PYG@bc=\relax \let\PYG@ff=\relax}
\def\PYG@tok#1{\csname PYG@tok@#1\endcsname}
\def\PYG@toks#1+{\ifx\relax#1\empty\else%
    \PYG@tok{#1}\expandafter\PYG@toks\fi}
\def\PYG@do#1{\PYG@bc{\PYG@tc{\PYG@ul{%
    \PYG@it{\PYG@bf{\PYG@ff{#1}}}}}}}
\def\PYG#1#2{\PYG@reset\PYG@toks#1+\relax+\PYG@do{#2}}

\expandafter\def\csname PYG@tok@s2\endcsname{\def\PYG@tc##1{\textcolor[rgb]{0.25,0.44,0.63}{##1}}}
\expandafter\def\csname PYG@tok@o\endcsname{\def\PYG@tc##1{\textcolor[rgb]{0.40,0.40,0.40}{##1}}}
\expandafter\def\csname PYG@tok@m\endcsname{\def\PYG@tc##1{\textcolor[rgb]{0.13,0.50,0.31}{##1}}}
\expandafter\def\csname PYG@tok@na\endcsname{\def\PYG@tc##1{\textcolor[rgb]{0.25,0.44,0.63}{##1}}}
\expandafter\def\csname PYG@tok@ss\endcsname{\def\PYG@tc##1{\textcolor[rgb]{0.32,0.47,0.09}{##1}}}
\expandafter\def\csname PYG@tok@c\endcsname{\let\PYG@it=\textit\def\PYG@tc##1{\textcolor[rgb]{0.25,0.50,0.56}{##1}}}
\expandafter\def\csname PYG@tok@cm\endcsname{\let\PYG@it=\textit\def\PYG@tc##1{\textcolor[rgb]{0.25,0.50,0.56}{##1}}}
\expandafter\def\csname PYG@tok@mo\endcsname{\def\PYG@tc##1{\textcolor[rgb]{0.13,0.50,0.31}{##1}}}
\expandafter\def\csname PYG@tok@w\endcsname{\def\PYG@tc##1{\textcolor[rgb]{0.73,0.73,0.73}{##1}}}
\expandafter\def\csname PYG@tok@nb\endcsname{\def\PYG@tc##1{\textcolor[rgb]{0.00,0.44,0.13}{##1}}}
\expandafter\def\csname PYG@tok@gh\endcsname{\let\PYG@bf=\textbf\def\PYG@tc##1{\textcolor[rgb]{0.00,0.00,0.50}{##1}}}
\expandafter\def\csname PYG@tok@k\endcsname{\let\PYG@bf=\textbf\def\PYG@tc##1{\textcolor[rgb]{0.00,0.44,0.13}{##1}}}
\expandafter\def\csname PYG@tok@kp\endcsname{\def\PYG@tc##1{\textcolor[rgb]{0.00,0.44,0.13}{##1}}}
\expandafter\def\csname PYG@tok@ge\endcsname{\let\PYG@it=\textit}
\expandafter\def\csname PYG@tok@nt\endcsname{\let\PYG@bf=\textbf\def\PYG@tc##1{\textcolor[rgb]{0.02,0.16,0.45}{##1}}}
\expandafter\def\csname PYG@tok@ow\endcsname{\let\PYG@bf=\textbf\def\PYG@tc##1{\textcolor[rgb]{0.00,0.44,0.13}{##1}}}
\expandafter\def\csname PYG@tok@vc\endcsname{\def\PYG@tc##1{\textcolor[rgb]{0.73,0.38,0.84}{##1}}}
\expandafter\def\csname PYG@tok@kd\endcsname{\let\PYG@bf=\textbf\def\PYG@tc##1{\textcolor[rgb]{0.00,0.44,0.13}{##1}}}
\expandafter\def\csname PYG@tok@bp\endcsname{\def\PYG@tc##1{\textcolor[rgb]{0.00,0.44,0.13}{##1}}}
\expandafter\def\csname PYG@tok@gd\endcsname{\def\PYG@tc##1{\textcolor[rgb]{0.63,0.00,0.00}{##1}}}
\expandafter\def\csname PYG@tok@gt\endcsname{\def\PYG@tc##1{\textcolor[rgb]{0.00,0.27,0.87}{##1}}}
\expandafter\def\csname PYG@tok@sb\endcsname{\def\PYG@tc##1{\textcolor[rgb]{0.25,0.44,0.63}{##1}}}
\expandafter\def\csname PYG@tok@sd\endcsname{\let\PYG@it=\textit\def\PYG@tc##1{\textcolor[rgb]{0.25,0.44,0.63}{##1}}}
\expandafter\def\csname PYG@tok@mb\endcsname{\def\PYG@tc##1{\textcolor[rgb]{0.13,0.50,0.31}{##1}}}
\expandafter\def\csname PYG@tok@s\endcsname{\def\PYG@tc##1{\textcolor[rgb]{0.25,0.44,0.63}{##1}}}
\expandafter\def\csname PYG@tok@sh\endcsname{\def\PYG@tc##1{\textcolor[rgb]{0.25,0.44,0.63}{##1}}}
\expandafter\def\csname PYG@tok@c1\endcsname{\let\PYG@it=\textit\def\PYG@tc##1{\textcolor[rgb]{0.25,0.50,0.56}{##1}}}
\expandafter\def\csname PYG@tok@kr\endcsname{\let\PYG@bf=\textbf\def\PYG@tc##1{\textcolor[rgb]{0.00,0.44,0.13}{##1}}}
\expandafter\def\csname PYG@tok@nc\endcsname{\let\PYG@bf=\textbf\def\PYG@tc##1{\textcolor[rgb]{0.05,0.52,0.71}{##1}}}
\expandafter\def\csname PYG@tok@gi\endcsname{\def\PYG@tc##1{\textcolor[rgb]{0.00,0.63,0.00}{##1}}}
\expandafter\def\csname PYG@tok@no\endcsname{\def\PYG@tc##1{\textcolor[rgb]{0.38,0.68,0.84}{##1}}}
\expandafter\def\csname PYG@tok@cp\endcsname{\def\PYG@tc##1{\textcolor[rgb]{0.00,0.44,0.13}{##1}}}
\expandafter\def\csname PYG@tok@sc\endcsname{\def\PYG@tc##1{\textcolor[rgb]{0.25,0.44,0.63}{##1}}}
\expandafter\def\csname PYG@tok@nd\endcsname{\let\PYG@bf=\textbf\def\PYG@tc##1{\textcolor[rgb]{0.33,0.33,0.33}{##1}}}
\expandafter\def\csname PYG@tok@mi\endcsname{\def\PYG@tc##1{\textcolor[rgb]{0.13,0.50,0.31}{##1}}}
\expandafter\def\csname PYG@tok@ne\endcsname{\def\PYG@tc##1{\textcolor[rgb]{0.00,0.44,0.13}{##1}}}
\expandafter\def\csname PYG@tok@si\endcsname{\let\PYG@it=\textit\def\PYG@tc##1{\textcolor[rgb]{0.44,0.63,0.82}{##1}}}
\expandafter\def\csname PYG@tok@nn\endcsname{\let\PYG@bf=\textbf\def\PYG@tc##1{\textcolor[rgb]{0.05,0.52,0.71}{##1}}}
\expandafter\def\csname PYG@tok@s1\endcsname{\def\PYG@tc##1{\textcolor[rgb]{0.25,0.44,0.63}{##1}}}
\expandafter\def\csname PYG@tok@kc\endcsname{\let\PYG@bf=\textbf\def\PYG@tc##1{\textcolor[rgb]{0.00,0.44,0.13}{##1}}}
\expandafter\def\csname PYG@tok@go\endcsname{\def\PYG@tc##1{\textcolor[rgb]{0.20,0.20,0.20}{##1}}}
\expandafter\def\csname PYG@tok@kt\endcsname{\def\PYG@tc##1{\textcolor[rgb]{0.56,0.13,0.00}{##1}}}
\expandafter\def\csname PYG@tok@mf\endcsname{\def\PYG@tc##1{\textcolor[rgb]{0.13,0.50,0.31}{##1}}}
\expandafter\def\csname PYG@tok@gu\endcsname{\let\PYG@bf=\textbf\def\PYG@tc##1{\textcolor[rgb]{0.50,0.00,0.50}{##1}}}
\expandafter\def\csname PYG@tok@ni\endcsname{\let\PYG@bf=\textbf\def\PYG@tc##1{\textcolor[rgb]{0.84,0.33,0.22}{##1}}}
\expandafter\def\csname PYG@tok@gs\endcsname{\let\PYG@bf=\textbf}
\expandafter\def\csname PYG@tok@sr\endcsname{\def\PYG@tc##1{\textcolor[rgb]{0.14,0.33,0.53}{##1}}}
\expandafter\def\csname PYG@tok@vi\endcsname{\def\PYG@tc##1{\textcolor[rgb]{0.73,0.38,0.84}{##1}}}
\expandafter\def\csname PYG@tok@nv\endcsname{\def\PYG@tc##1{\textcolor[rgb]{0.73,0.38,0.84}{##1}}}
\expandafter\def\csname PYG@tok@nf\endcsname{\def\PYG@tc##1{\textcolor[rgb]{0.02,0.16,0.49}{##1}}}
\expandafter\def\csname PYG@tok@gp\endcsname{\let\PYG@bf=\textbf\def\PYG@tc##1{\textcolor[rgb]{0.78,0.36,0.04}{##1}}}
\expandafter\def\csname PYG@tok@se\endcsname{\let\PYG@bf=\textbf\def\PYG@tc##1{\textcolor[rgb]{0.25,0.44,0.63}{##1}}}
\expandafter\def\csname PYG@tok@err\endcsname{\def\PYG@bc##1{\setlength{\fboxsep}{0pt}\fcolorbox[rgb]{1.00,0.00,0.00}{1,1,1}{\strut ##1}}}
\expandafter\def\csname PYG@tok@gr\endcsname{\def\PYG@tc##1{\textcolor[rgb]{1.00,0.00,0.00}{##1}}}
\expandafter\def\csname PYG@tok@cs\endcsname{\def\PYG@tc##1{\textcolor[rgb]{0.25,0.50,0.56}{##1}}\def\PYG@bc##1{\setlength{\fboxsep}{0pt}\colorbox[rgb]{1.00,0.94,0.94}{\strut ##1}}}
\expandafter\def\csname PYG@tok@mh\endcsname{\def\PYG@tc##1{\textcolor[rgb]{0.13,0.50,0.31}{##1}}}
\expandafter\def\csname PYG@tok@nl\endcsname{\let\PYG@bf=\textbf\def\PYG@tc##1{\textcolor[rgb]{0.00,0.13,0.44}{##1}}}
\expandafter\def\csname PYG@tok@vg\endcsname{\def\PYG@tc##1{\textcolor[rgb]{0.73,0.38,0.84}{##1}}}
\expandafter\def\csname PYG@tok@il\endcsname{\def\PYG@tc##1{\textcolor[rgb]{0.13,0.50,0.31}{##1}}}
\expandafter\def\csname PYG@tok@sx\endcsname{\def\PYG@tc##1{\textcolor[rgb]{0.78,0.36,0.04}{##1}}}
\expandafter\def\csname PYG@tok@kn\endcsname{\let\PYG@bf=\textbf\def\PYG@tc##1{\textcolor[rgb]{0.00,0.44,0.13}{##1}}}

\def\PYGZbs{\char`\\}
\def\PYGZus{\char`\_}
\def\PYGZob{\char`\{}
\def\PYGZcb{\char`\}}
\def\PYGZca{\char`\^}
\def\PYGZam{\char`\&}
\def\PYGZlt{\char`\<}
\def\PYGZgt{\char`\>}
\def\PYGZsh{\char`\#}
\def\PYGZpc{\char`\%}
\def\PYGZdl{\char`\$}
\def\PYGZhy{\char`\-}
\def\PYGZsq{\char`\'}
\def\PYGZdq{\char`\"}
\def\PYGZti{\char`\~}
% for compatibility with earlier versions
\def\PYGZat{@}
\def\PYGZlb{[}
\def\PYGZrb{]}
\makeatother

\renewcommand\PYGZsq{\textquotesingle}

\begin{document}

\maketitle
\tableofcontents
\phantomsection\label{index::doc}



\chapter{Introduction}
\label{intro:bienvenue-sur-le-travail-ecrit}\label{intro::doc}\label{intro:introduction}
Notre monde actuel devient chaque année plus proche de la technologie et se veut
devenir petit à petit un monde connecté. Ceci est sans doute une des raisons
pour lesquelles ce séminaire de développement d'une platforme de formation
en ligne a été lancé. En effet, grâce à l'informatique, nous pouvons élaborer
des techniques d'enseignement qui n'étaient pas disponibles il y a quelques
années, ce qui sous-entend que la pédagogie de l'époque n'était pas la même qu'aujourd'hui et que
celle-ci évolue au fil du temps. De ce fait, il nous faut constamment se mettre à jour si l'on
veut explorer d'autres moyens d'étude et d'enseignement. Je vais, dans ce travail,
programmer une application web en version mobile qui permettra aux élèves de
communiquer avec leurs professeurs. Ils auront l'occasion de le faire non pas par l'échange de messages mais par le
téléversement d'images. Le but premier de mon travail est de fournir au professeur
un outil grâce auquel il pourra par exemple demander à ses élèves de faire un
exercice et le photographier avec leurs smartphone avant de l'envoyer sur un serveur auquel celui-là
aura accès et pourra donc voir les différentes résolutions du travail de ses élèves.
A travers cette technique, le gain de temps en classe pour lancer une correction
est donc flagrant car le professeur connait à l'avance le résultat de ses élèves
et peut adapter son cours en fonction du résultat de ses élèves. Afin d'arriver a ce dessein je vais d'abord
présenter les particularités du développement mobile qui est spécifique aux
appareils portables. Ensuite je ferai la présentation de jQuery Mobile, cette
fameuse bibliothèque qui sera au coeur de ma problèmatique. Et Ce travail sera  finalement conclu
par l'explication et la présentation des fonctionnalités de mon application.


\chapter{Particularités du développement mobile}
\label{Particularit_xe9s::doc}\label{Particularit_xe9s:particularites-du-developpement-mobile}
En élaborant une interface mobile, il est indispensable de se demander
si ce type de travail présente des spécificités auxquelles on ne fait pas face en
développant une version bureau.

Effectivement, on se rend vite compte par l'utilisation d'un smartphone que la
plupart des interfaces utilisateurs subissent une refonte. On observe notammant
des boutons plus gros, mieux adaptés au toucher sur les écrans tactiles.
Ceci est une des grandes forces de jQuery Mobile car il propose une mise en forme de tous
les constituants d'une page web très agréable à l'utilisation sur des appareils mobiles.
De plus, quand on utilise un smartphone, on se retrouve souvent sur un réseau
mobile à débit ou volume de données limitées. De ce fait, il serait favorable à
l'utilisateur de l'application et au développeur de fournir des pages peu gourmandes en données.
Dans mon application, il serait intéressant d'intégrer un outil de compression d'images.
L'utilisateur pourra ainsi, lorsqu'il charge une liste d'images, avoir
un aperçu rapide à toutes celles qui ont été compressées puis téléversées, et par après accéder
à un détail si tel en est le souhait afin de bénéficier d'une meilleure qualité.
Finalement, quand on navigue sur un appareil mobile, on a l'occasion d'exécuter
des gestes ou `évènements' qui ne se font pas sur un écran avec lequel on interagit
avec un clavier ou une souris. Par exemple, le fait de glisser de gauche à droite
pour ouvrir une extension de la page ou encore l'appui prolongé sur un élément
qui peut être comparé au double-clic sur une interface standard.

En fin de compte, le développement mobile comporte diverses facettes auxquelles il faut
s'adapter afin de fournir un produit intéressant et utilisable sur un mobile. Ces éléments, présentés
parmi d'autres, donnent en partie un exemple de ce jQuery Mobile peut gérer.
Lors de la construction des pages web de mon application, je prendrai soin
qu'elles soient bien aptes à être utilisées sur mobile et les testerai sur divers smartphones
dotés de diverses tailles d'écrans et specificités techniques. Ceci me permettra d'être assuré que la
plupart des personnes puissent venir à utiliser cette application.


\chapter{jQuery Mobile et Bootstrap}
\label{Diff_xe9rence_jQM_boot::doc}\label{Diff_xe9rence_jQM_boot:jquery-mobile-et-bootstrap}

\section{Fonctionnement général de jQuery Mobile}
\label{Diff_xe9rence_jQM_boot:fonctionnement-general-de-jquery-mobile}
La majeure partie de jQuery Mobile se joue dans le balisage de son code html.
Dans la mesure où l'on définit si le contenu d'un balisage:

Du contenu:

\begin{Verbatim}[commandchars=\\\{\}]
\PYG{n+nt}{\PYGZlt{}div}\PYG{n+nt}{\PYGZgt{}}contenu\PYG{n+nt}{\PYGZlt{}/div\PYGZgt{}}
\end{Verbatim}

Un lien:

\begin{Verbatim}[commandchars=\\\{\}]
\PYG{n+nt}{\PYGZlt{}a} \PYG{n+na}{href=}\PYG{l+s}{\PYGZdq{}\PYGZsh{}\PYGZdq{}}\PYG{n+nt}{\PYGZgt{}}lien\PYG{n+nt}{\PYGZlt{}/a\PYGZgt{}}
\end{Verbatim}

et pourquoi pas un bouton:

\begin{Verbatim}[commandchars=\\\{\}]
\PYG{n+nt}{\PYGZlt{}button}\PYG{n+nt}{\PYGZgt{}}bouton\PYG{n+nt}{\PYGZlt{}/button\PYGZgt{}}
\end{Verbatim}

deviendra une page, une boîte de dialogue ou encore une liste entre autres. Pour un petit test,
faisons l'exemple avec ces trois morceaux de code. Ainsi en ajoutant des classes
telles que ui-content pour le contenu:

\begin{Verbatim}[commandchars=\\\{\}]
\PYG{n+nt}{\PYGZlt{}div} \PYG{n+na}{class=}\PYG{l+s}{\PYGZdq{}ui\PYGZhy{}content\PYGZdq{}}\PYG{n+nt}{\PYGZgt{}}contenu\PYG{n+nt}{\PYGZlt{}/div\PYGZgt{}}
\end{Verbatim}

Les scripts jQuery Mobile viendront s'appliquer là-dessus et considéreront ceci
comme le contenu d'une page et appliqueront le code css nécessaire.
Pareil pour le lien et le bouton qui suivent. Nous pouvons y ajouter la classe
`ui-btn' qui dira aux scripts jQuery Mobile d'appliquer de code css nécessaire
pour avoir l'allure d'un bouton.

\begin{Verbatim}[commandchars=\\\{\}]
\PYG{n+nt}{\PYGZlt{}a} \PYG{n+na}{href=}\PYG{l+s}{\PYGZdq{}\PYGZsh{}\PYGZdq{}} \PYG{n+na}{class=}\PYG{l+s}{\PYGZdq{}ui\PYGZhy{}btn\PYGZdq{}}\PYG{n+nt}{\PYGZgt{}}lien\PYG{n+nt}{\PYGZlt{}/a\PYGZgt{}}
\PYG{n+nt}{\PYGZlt{}button} \PYG{n+na}{class=}\PYG{l+s}{\PYGZdq{}ui\PYGZhy{}btn\PYGZdq{}}\PYG{n+nt}{\PYGZgt{}}bouton\PYG{n+nt}{\PYGZlt{}/button\PYGZgt{}}
\end{Verbatim}

Pour l'utilisation de jQuery Mobile, il est donc nécessaire de travailler avec ce balisage
qui permettra à la bibliotheque d'interpréter le code et d'y appliquer les
attributs et la mise en page nécessaire avec une allure très agréable à
l'utilisation tactile et très bien adaptée aux écrans de taille plutôt réduite que l'on
peut retrouver sur un smartphone standard voire sur une tablette de petite taille.


\section{Fonctionnement général de Bootstrap}
\label{Diff_xe9rence_jQM_boot:fonctionnement-general-de-bootstrap}
Bootstrap ne sera pas au coeur de ce travail mais il est intéressant de comparer
ces bibliothèques qui peuvent paraître proches mais qui finalement offrent un rendu
relativement opposé. Le principe de bootstrap est basé non pas sur le balisage
comme jQuery Mobile mais sur une sorte de grille. Cette grille sera composée d'un
certain nombre de colonnes et de lignes où l'utilisateurs pourra positionner les
éléments qu'il désire afficher sur sa page. Et c'est lors du changement de support
que l'on peut observer toute la magie de bootstrap car cette grille s'adapte elle-même
à l'écran. Ainsi le contenu se dit ``responsive'', soit ``qui s'adapte''. Par exemple,
un barre de navigation initialement placée sur le côté gauche verticalement si l'on consulte
le site sur un écran large pourrait se retrouver horizontalement sur un écrant plus
étroit au dessus du contenu principal.


\section{Similitudes et différences}
\label{Diff_xe9rence_jQM_boot:similitudes-et-differences}
Ces deux bibliothèques se ressemblent dans la mesure où elles me seraient toutes
les deux utiles afin de créer une interface mobile pour ma future application.
Elles proposent un affichage qui est facilement utilisable dans des conditions
que l'on ne retrouve pas sur un ordinateur de bureau et auxquelles on ne pense pas
forcément au cours du développement. Ceci nous permet donc une interface utilisateur
adaptée aux besoins d'une personne utilisant cette application mobile. Les deux
présentent également un très bon système pour modifier les thèmes et ainsi
apporter un côté ludique ou encore plus agréable.
Par contre Bootstrap se concentre vraiment sur une allure du site qui s'adapte
aux différents formats d'écrans tandis que jQuery Mobile est plutôt dans l'optique
de proposer un rendu qui lui est entièrement consacré au mobile en se souciant peu
de l'affichage sur une plus grand écran. Malgré cela, on peut dire que l'affichage
mobile sur un grand écran n'est pas désagréable notammant si l'on dispose d'un écran
tactile mais ce n'est pas le meilleur que l'on puisse avoir.


\chapter{Présentation de jQuery Mobile}
\label{presentation_jQM::doc}\label{presentation_jQM:presentation-de-jquery-mobile}

\section{Pourquoi avoir choisi une bibliothèque de ce genre ?}
\label{presentation_jQM:pourquoi-avoir-choisi-une-bibliotheque-de-ce-genre}
JQuery Mobile est une bibliothèque qui facilite grandement le développement web.
Elle a été mise sur pieds par la même maison que la si fameuse bibliothèque
Javascript également nommée jQuery. Dans ma partie du projet, il est avant tout
question de développement web mobile. Lorsque l'on parle d'un projet mobile, trois
grandes approches sont envisageables. On peut aussi bien partir sur le principe
de développer une application native, adaptée à un système d'exploitation spécifique.
Cette possibilité a l'avantage d'être très optimisée et performante mais nous
contraint à développer une application pour chaque support que nous viendrons
à utiliser. On peut aussi développer une application dite hybride, à mi-chemin entre le natif et
le site web version mobile. La plupart du temps, celle-ci est codée en Javascript,
html et css qui sera ensuite compilée pour offrir un rendu à l'allure native mais
plutôt basée sur un code qui se rapproche du développement web. Cette alternative
a l'avantage d'être multi-platforme mais un peu moins optimisées que les
applications natives. Et finalement, je présente la voie que je vais suivre,
celle du développement web en version mobile grâce à l'utilisation de cette
bibliothèque jQuery Mobile. J'ai adopté ce choix en fonction de mes connaissances
en programmation et des possiblités qu'il offre. Du fait que les pages internet
codées à l'aide de jQuery Mobile s'affiche sur un navigateur, il devient
évident que tous les supports dôtés d'un accès internet puisse profiter de cette interface,
et cela quelque soit le système d'exploitation utilisé.


\section{Existe-t-il des concurents à jQuery Mobile ?}
\label{presentation_jQM:existe-t-il-des-concurents-a-jquery-mobile}
Comme dans la plus grande majorité des domaines, la concurence est de
la partie et cela même quand il s'agit de programmation. On peut par exemple citer:
Kendo UI, ChocolateChip-UI ou encore bootstrap d'une certaine façon, nous en
reparlerons plus tard dans ce travail. Parmi ces concurents, on trouve certains
qui disposent de fonctionnalités inédites comme la géolocalisation, des thèmes
s'inspirant des surcouches utilisateurs connues et encore des petites différences
qui peuvent influencer notre choix de bibliothèque mais qui en fin de compte
toutes offre un résultat similaire.


\section{Le choix de jQuery Mobile plutôt que d'un autre concurent}
\label{presentation_jQM:le-choix-de-jquery-mobile-plutot-que-d-un-autre-concurent}
Après avoir observé différentes bibliothèques et comparé leurs possibilités,
ceci ne me paraissait pas un choix très important et j'ai préféré resté sur
l'idée de départ qui est jQuery Mobile. J'ai facilement associé celle-ci
à la bibliothèque Javascript déjà existante du même nom dont la renommée n'est pas
à remettre en question. JQuery Mobile et aussi indirectement jQuery seront au
centre de mon travail de recherche et pratique.


\chapter{Views.py et urls.py}
\label{views_urls::doc}\label{views_urls:views-py-et-urls-py}

\chapter{Models.py}
\label{models::doc}\label{models:models-py}

\chapter{Templates}
\label{templates::doc}\label{templates:templates}

\chapter{Mise en pratique de mon application}
\label{pratique::doc}\label{pratique:mise-en-pratique-de-mon-application}

\chapter{Ressources utilisées pour mon TM}
\label{ressources::doc}\label{ressources:ressources-utilisees-pour-mon-tm}\begin{itemize}
\item {} 
\href{https://docs.djangoproject.com/fr/1.7/}{https://docs.djangoproject.com/fr/1.7/}

\item {} 
\href{http://openclassrooms.com/courses/apprenez-a-creer-votre-site-web-avec-html5-et-css3}{http://openclassrooms.com/courses/apprenez-a-creer-votre-site-web-avec-html5-et-css3}

\item {} 
\href{http://openclassrooms.com/courses/dynamisez-vos-sites-web-avec-javascript}{http://openclassrooms.com/courses/dynamisez-vos-sites-web-avec-javascript}

\item {} 
\href{http://api.jquery.com/}{http://api.jquery.com/}

\item {} 
\href{http://api.jquerymobile.com/}{http://api.jquerymobile.com/}

\end{itemize}



\renewcommand{\indexname}{Index}
\printindex
\end{document}
